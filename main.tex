\documentclass[10pt,a4paper,sans]{moderncv}        % possible options include font size ('10pt', '11pt' and '12pt'), paper size ('a4paper', 'letterpaper', 'a5paper', 'legalpaper', 'executivepaper' and 'landscape') and font family ('sans' and 'roman')


% moderncv themes
\moderncvstyle{classic}                             % style options are 'casual' (default), 'classic', 'oldstyle' and 'banking'
\moderncvcolor{orange}                               % color options 'blue' (default), 'orange', 'green', 'red', 'purple', 'grey' and 'black'

% character encoding
\usepackage[utf8]{inputenc}                       % if you are not using xelatex ou lualatex, replace by the encoding you are using

\newboolean{long}
\setboolean{long}{false}

% adjust the page margins
\usepackage[scale=0.75, top=50pt, bottom=60pt]{geometry}
%\setlength{\hintscolumnwidth}{3cm}                % if you want to change the width of the column with the dates
%\setlength{\makecvtitlenamewidth}{10cm}           % for the 'classic' style, if you want to force the width allocated to your name and avoid line breaks. be careful though, the length is normally calculated to avoid any overlap with your personal info; use this at your own typographical risks...

% personal data
\renewcommand*{\namefont}{\fontsize{24}{20}\mdseries\upshape}
\renewcommand*{\emailsymbol}{}
\name{Jonathan}{\mbox{Sid-Otmane}}

\title{\mbox{Spécialisé en Système Distribué}}
%\title{Docteur en Informatique}
%\title{Docteur en Système Distribué}
\address{Appt. 711\\162 Av. Paul Vaillant Couturier}{75014, Paris}% optional, remove / comment the line if not wanted; the "postcode city" and and "country" arguments can be omitted or provided empty
\phone[mobile]{00 33 6 11 37 02 97}                   % optional, remove / comment the line if not wanted
\social[linkedin][https://www.linkedin.com/in/Jonathan-Sid-Otmane/]{Jonathan Sid-Otmane}
\email{Jonathan.SidOtmane@gmail.com}                               % optional, remove / comment the line if not wanted
\extrainfo{30 ans, Français}
\photo[64pt][0pt]{Jonathan}                       % optional, remove / comment the line if not wanted; '64pt' is the height the picture must be resized to, 0.4pt is the thickness of the frame around it (put it to 0pt for no frame) and 'picture' is the name of the picture file
%\quote{\mbox{PhD} }                           % optional, remove / comment the line if not wanted

%----------------------------------------------------------------------------------
%            content
%----------------------------------------------------------------------------------
\begin{document}
%-----       resume       ---------------------------------------------------------
\makecvtitle


%Chercheur en Système Distribués je souhaite me consacrer à des problématiques 


%% Accroche


\section{Expériences Professionnelles et de Recherche}
%\subsection{Vocational}
\cventry{2017-2021}{Doctorat CIFRE}{Orange Labs \& LIP6 Sorbonne Université, advisor Marc Shapiro}{Paris}{}{
    \begin{itemize}
        %\item Membre de l'équipe DELYS au LIP6, qui travaille notamment sur le développement de la base de données expérimentale  Antidote.
            %\item Membre de l'équipe IRON chez Orange Labs, consacrée à la transition de infrastructures télécoms vers 
        \item État de l'art sur la \textbf{cohérence} des protocoles de \textbf{réplication} dans les systèmes \textbf{distribués} et sur les méthodes de déploiement d'\textbf{infrastructure virtualisées}. 
            % (La \textbf{base de données} de recherche Antidote)
        \item Mise en évidence de conflits entre la spécification de l'infrastructure 5G et le théorème de \textbf{CAP}. Étude du cycle de vie des données dans le réseau après une étape d'analyse et de simplification de la spécification.
            \item  Proposition d'invariants nécessaires pour assurer un bon fonctionnement du réseau et de mécanismes compatibles avec CAP et suffisants pour les maintenir. 
            \item Modification d'un simulateur à évènements discrets pour les évaluer dans un scénario semblable à un déploiement 5G à grande échelle.
            \item Une partie de ces résultats ont été publiés à ICIN2020: Data Consistency in the 5G Specification.
    \end{itemize} }

\cventry{}{Chargé d'enseignement}{\mbox{Sorbonne} Université}{Paris}{}{
Attaché Temporaire d'Éducation et de Recherche (ATER):
    %Chargé d'enseignement à temps plein dans les UEs:
    %Chargé d'enseignement à temps plein et chercheur au sein de l'équipe DELYS. 
Introduction à la programmation objet en Java (Licence), 
Fondements des systèmes d'exploitation (Licence),
Introduction à la programmation Système en Shell (Licence),
et \textbf{Programmation Système Répartie et Concurrente} (Master).
}

\cventry{2016 (5 mois)}{Stage de Recherche}{Université de Brown}{Providence, RI. USA}{}{
    Étude de la performance d'un prototype du serveur M7 d'Oracle. 
    Création de workflows et de micro-benchmarks pour mesurer la performance des coprocesseurs DAX et les comparer à l'état de l'art.
    Contribution à un prototype de système de gestion de bases de données qui tire parti de stratégies d'exécution hybrides.
}

% \cventry{2016 (5 mois)}{Stage de Recherche}{Université de Brown}{Providence, RI. USA}{}{
    % Mesure de la performance des coprocesseurs DAX de l'architecture M7 d'Oracle via la création de workflows et de micro-benchmarks. Création d'un prototype de  système de gestion de base de données qui tire parti de ces composants.
    % %Le papier SiliconDB repose en partie sur ce travail.
% }

		\cventry{2015 (3 mois)}{Stage de Recherche}{Laboratoire d'Informatique de Paris 6, LIP6}{Université Pierre et Marie Curie}{}{
            Diminuer les coûts de communication entre machines virtuelles co-localisées en manipulant les adresses virtuelles dans la MMU de  l'hyperviseur.
            Première expérience de programmation dans le Kernel Linux avec KVM et Qemu.
}

		% \cventry{2015 (3 mois)}{Stage de Recherche}{Laboratoire d'Informatique de Paris 6, LIP6}{Université Pierre et Marie Curie}{}{
 % Étude des mécanismes internes de virtualisation et de gestion de la mémoire dans Linux et KVM. Création  et évaluation d'un mécanisme de communication entre machines paravirtualisées dans Qemu.
			% % \begin{itemize}
				% % \item État de l'art des solutions de communication entre machines virtuelles et hyperviseur.
				% % \item Étude des mécanismes internes de virtualisation et de gestion de la mémoire dans Linux et KVM.
				% % \item Création d'un système de communication entre machines paravirtualisées dans Qemu et évaluation face à des tests de performance artificiels.
			% % \end{itemize}
% }

%\cventry{year--year}{Job title}{Employer}{City}{}{Description}


\section{Formation}
\cventry{2021}{Doctorat}{Étude des contraintes de cohérence des données dans la 5G, appliquée aux limitations d'usage de ressources dans les slices réseau}{Sorbonne Université \& Orange Labs}{}{}
\cventry{2016}{Master Système et Applications Réparties (SAR)}{UPMC}{Paris VI}{}{
    %Conception et développement de systèmes répartis, parallèles, pair à pair, large échelle. Conception des systèmes et utilisation des couches basses. 
    Systèmes distribués, Algorithmie Répartie, et Virtualisation
        }
\cventry{2014}{Licence Informatique}{Université Pierre et Marie Curie (UPMC)}{Paris VI}{}{}

\ifthenelse{\boolean{long}}{
\section{Enseignement}
\cventry{}{Programmation Système Répartie et Concurrente (MU4IN400)}{}{Master}{\mbox{3 $\times$ 40hrs}}{
    Introduction à la programmation système en C++. Concurrence (fork et thread), communication inter-processus (signaux, pipes, sockets, mémoire partagée), et syncrhonisation (mutex, conditions, sémaphores).
    }
\cventry{}{Fondements des systèmes d'exploitation (LU3IN010)}{}{Licence}{\mbox{40hrs}}{
    Stratégies d'ordonnancement,
    interruptions,
    mémoire virtuelle et espace d'adressage,
    stratégies de swap,
    systèmes de fichiers,
    parallélisme et synchronisation (Sémaphores POSIX).
}
\cventry{}{Introduction à la programmation Système en Shell (LU2IN020)}{}{Licence}{\mbox{2 $\times$ 40hrs}}{
    Programmation en Bash, 
    environnement Linux,
    expressions régulières,
    cycle de vie d'un processus,
    présentation du Kernel (modes U et S),
    agencement de la mémoire des processus,
    fonctionnement des appels système,
    parallélisme et synchronisation (processus en tâche de fond et wait), signaux.
    }
\cventry{}{Introduction à la programmation objet en Java (LU2IN002)}{}{Licence}{\mbox{40hrs}}{
    Modélisation de problèmes en POO, classes, héritage, interfaces. Introduction à la librairie standard.
    }
}{}

\section{Compétences Techniques}


\cvitem{Générales}{
    Système distribué,
    Programmation Concurrente et Répartie,
    Programmation Objet,
    Protocoles de Cohérence,
    Bases de donnés réparties,
    Infrastructure 5G,
    Algorithmie,
    Linux Kernel,
    Virtualisation,
    CICD,
    SQL,
    }

\cvitem{Outils}{
        Git,
        Vim,
        Environnement UNIX,
        VSCode
}
\cvitem{Langages}{\textbf{C/C++}, \textbf{Java}, Bash, Python, Rust}

% \section{Compétences}
% \cvitem{}{Maîtrise des paradigmes de programmation objet, concurrent, et réparti. Dévelopment dans un environnement UNIX avec l'outil Git et le CICD. .; des Design Patterns ainsi que du développement dans un environnement POSIX.}
% % \cvitem{}{Connaissances des mécanismes internes de Linux et de l'hyperviseur KVM.}
% \cvitem{}{Connaissance avancée des langages C et Java ainsi que de l'outil de gestion des sources Git.}
% \cvitem{}{Dévelopment dans un environnement UNIX avex Git et CICD. Maîtrise des paradigmes de programmation objet, concurrent, et réparti.}


\section{}
\cvitemwithcomment{Français}{Langue maternelle}{}
\cvitemwithcomment{Anglais}{Courant}{}
%\cvitemwithcomment{Allemand}{Inexistant}{}
\cvitemwithcomment{}{Permis B}{}

%\section{Interests}
%\cvitem{hobby 1}{Description}


% Publications from a BibTeX file without multibib
%  for numerical labels: \renewcommand{\bibliographyitemlabel}{\@biblabel{\arabic{enumiv}}}% CONSIDER MERGING WITH PREAMBLE PART
%  to redefine the heading string ("Publications"): \renewcommand{\refname}{Articles}

% \nocite{*}
% \bibliographystyle{plain}
% \bibliography{publications}                        % 'publications' is the name of a BibTeX file

% Publications from a BibTeX file using the multibib package
%\section{Publications}
%\nocitebook{book1,book2}
%\bibliographystylebook{plain}
%\bibliographybook{publications}                   % 'publications' is the name of a BibTeX file
%%\nocitemisc{misc1,misc2,misc3}
%\bibliographystylemisc{plain}
%\bibliographymisc{publications}                   % 'publications' is the name of a BibTeX file

%%%\clearpage
%-----       letter       ---------------------------------------------------------
% recipient data
\recipient{Company Recruitment team}{Company, Inc.\\123 somestreet\\some city}
\date{Mercredi 30 Septembre 2015}
\opening{Madame, Monsieur}
\closing{Dans l'attente de votre réponse positive,}
%\enclosure[Attached]{curriculum vit\ae{}}          % use an optional argument to use a string other than "Enclosure", or redefine \enclname
%%%\makelettertitle

%%%Lorem ipsum dolor sit amet, consectetur adipiscing elit. Duis ullamcorper neque sit amet lectus facilisis sed luctus nisl iaculis. Vivamus at neque arcu, sed tempor quam. Curabitur pharetra tincidunt tincidunt. Morbi volutpat feugiat mauris, quis tempor neque vehicula volutpat. Duis tristique justo vel massa fermentum accumsan. Mauris ante elit, feugiat vestibulum tempor eget, eleifend ac ipsum. Donec scelerisque lobortis ipsum eu vestibulum. Pellentesque vel massa at felis accumsan rhoncus.

%%%Duis sit amet magna ante, at sodales diam. Aenean consectetur porta risus et sagittis. Ut interdum, enim varius pellentesque tincidunt, magna libero sodales tortor, ut fermentum nunc metus a ante. Vivamus odio leo, tincidunt eu luctus ut, sollicitudin sit amet metus. Nunc sed orci lectus. Ut sodales magna sed velit volutpat sit amet pulvinar diam venenatis.


 %%%%\makeletterclosing

%\clearpage\end{CJK*}                              % if you are typesetting your resume in Chinese using CJK; the \clearpage is required for fancyhdr to work correctly with CJK, though it kills the page numbering by making \lastpage undefined
\end{document}

